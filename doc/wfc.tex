\documentclass[12pt]{article}
\title{WFC - wire format compiler}
\author{Thomas Maier-Komor}
\date{September 2018}

\begin{document}

\maketitle

\section{About}
WFC, the wire format compiler provides the infrastructure for
serializing and deserializing objects to binary streams independently of
the software and hardware platform. The generated code is especially
optimized for embedded systems, and serveral options provide means to
tune the code regarding specific aspects. Like this the code can be
tailored for running on a variety of systems from 8 bits to 64 bits.

The description language for defining message classes is a
language-extension of Protocol Buffers. Therefore, WFC is able to work
with {\tt .proto} files that were written for Protocal Buffers.

\section{Data Serialization}
Data serialization works as in Protocol Buffers V2 and later, but provides
extensions for embedded systems.

Integer data is either stored directly in little endian format as fixed length
block or as a varialbe length byte sequence. Variable length byte sequence
stores 7 bits in one byte beginning with the least significant bit. The highest
bit indicates if there are more bytes. Once all more siginicant bits are zero,
bit 7 is set to zero to indicate the end of the variable length integer
sequence.


\subsection{Variable Length Encoding}

The tables below show how a sample 32-bit word is encoded into a serialized
data stream using variable length encoding.

\subsubsection*{Sample 32-bit word {\tt 0xdcba} to be encoded}
\begin{tabular}{|l|c|c|c|c|}
\hline
	Bit&31-24&23-16&15-8&7-0\\
\hline
	little endian address & {3} & {2} & {1} & {0} \\
\hline
	big-endian address & {0} & {1} & {2} & {3} \\
\hline
	hex value& {0xd} & {0xc} & {0xb} & {0xa} \\
\hline
	binary & {\tt 0b00001101} & {\tt 0b00001100} & {\tt 0b00001011} & {\tt 0b00001010}\\
\hline
\end{tabular}

\subsubsection*{Sequentially Encoded Sample}
\begin{tabular}{|l|c|c|c|c|}
\hline
	serialized data & {bit 0..6} & {bit 7..13} & {bit 14..20} &  {bit 21..27} \\
\hline
	extension bit & $1<<7$ & $1<<7$ & $1<<7$ & $0<<7$ \\
\hline
	extracted word values & {\tt 0x6} & {\tt 0x16} &  {\tt 0x30} &  {\tt 0x78} \\
\hline
	or'ed hex value & {\tt 0x86} & {\tt 0x96} & {\tt 0xb0} & {\tt 0x78} \\
\hline
	binary values &{\bf 1}0001010 & {\bf 1}0010110 & {\bf 1}0110000 & {\bf 0}1101000\\
\hline
\end{tabular}


\subsection{Encoding Negative Values}
Negative values can be encoded using two different methods:
\begin{itemize}
	\item[regular encoding] Regular encoding is used for data types {\tt
		int, int16, int32, int64}. For these data types nothing special
		is done, so all negative values will consume 10 bytes in a
		sequential data stream, as all higher value bits are {\tt 1}.
		For forward compatibility reasons the variable length encoding
		is always performed as for a 64bit word.
	\item[signed encoding] Signed encoding is used for data types {\tt
		signed, sint16, sint32, sint64}. For signed encoding the most
		significant bit is shifted into bit 0, all other bits are
		shifted left by one, and xor'ed by 1 if the most significant
		bit is {\tt 1}.
\end{itemize}



\section{Data Types}
WFC offers several data types, some of which are implementation dependent. The
in memory representation and serialization of some data types may depend on
options. Please refer to the description of options, to learn the details.

\subsection{Data Typese that are compatible to proto2}
\begin{center}
\begin{tabular}{|l|c|l|}
\hline
	data type & valid range & encoding\\
\hline
	bool	& false,true			& 1 Byte, varint\\
	enum	& enumeration			& 1-10 Bytes, varint\\
	int32	& SINT32\_MIN..SINT32\_MAX	& 1-10 Bytes, varint\\
	sint32	& SINT32\_MIN..SINT32\_MAX	& 1-5 Bytes, varint\\
	fixed32	& 0..UINT32\_MAX		& 4 Bytes, fixed length\\
	sfixed32& SINT32\_MIN..SINT32\_MAX	& 4 Bytes, fixed length\\
	int64	& SINT64\_MIN..SINT64\_MAX	& 1-10 Bytes, varint\\
	sint64	& SINT64\_MIN..SINT64\_MAX	& 1-10 Bytes, varint\\
	fixed64	& 0..UINT64\_MAX		& 8 Bytes, fixed length\\
	sfixed64& SINT64\_MIN..SINT64\_MAX	& 8 Bytes, fixed length\\
	float	& FLT\_MIN..FLT\_MAX		& 4 Bytes, fixed legnth\\
	double	& DBL\_MIN..DBL\_MAX		& 8 Bytes, fixed legnth\\
	string	& byte array			& varint length prefix followed by data\\
	bytes	& byte array			& varint length prefix followed by data\\
\hline
\end{tabular}
\end{center}

\subsection{New Data Types}
\begin{center}
\begin{tabular}{|l|c|l|}
\hline
	data type & valid range & encoding\\
\hline
	int	& depends on {\tt intsize}	& 1-10 Bytes, varint\\
	signed	& depends on {\tt intsize}	& 1-10 Bytes, varint\\
	unsigned& depends on {\tt intsize}	& 1-10 Bytes, varint\\
	uint8	& 0..255			& 1-2 Bytes, varint\\
	int8	& -128..127			& 1-10 Bytes, varint\\
	sint8	& -128..127			& 1-2 Bytes, varint\\
	fixed8	& 0..255			& 1 Byte, fixed length\\
	sfixed8	& -128..127			& 1 Byte, fixed length\\
	int16	& SINT16\_MIN..SINT16\_MAX	& 1-10 Bytes, varint\\
	sint16	& SINT16\_MIN..SINT16\_MAX	& 1-3 Bytes, varint\\
	fixed16	& 0..UINT16\_MAX		& 2 Bytes, fixed length\\
	sfixed16& SINT16\_MIN..SINT16\_MAX	& 2 Bytes, fixed length\\
\hline
\end{tabular}
\end{center}


\subsection{Repeated Data Types}
Repeated Data Types are implemented using the STL class {\tt std::vector} per
default.

The binary option {\tt packed} may be used for a more efficient data
serialization. Per default every element of a repeated variable is serialized
with its own tag. For packed repeated data only one tag is written and the data
is encoded with length prefix. This usually requires less memory.

If option {\tt arraysize} is set, the element is implemented as a
fixed size array using the {\tt array} class in {\tt include/array.h}. This may
be preferable on systems with limited memory, when the maximum size of the
array is known upfront.

\subsection{Strings and Byte Arrays}
The data types {\tt string} and {\tt bytes} are implemented using STL's {\tt
std::string} template class per default. Using options {\tt stringtype} and
{\tt bytestype} a different type for implementation can be specified.

Set option {\tt stringtype} or {\tt bytestype} to the appropriate {\tt
typename}. For {\tt stringtypes} an additional valid option is {\tt C} to
implement as a C-String ({\tt const char *}). Be aware that for C-Strings there
will never be a copy performed and no memory management is active. I.e. the
pointed to data must stay valid as long as the message object is valid.
Additionally, the C-String must not contain null bytes and must be terminated
by a null byte, just like the C convention requires.

For deserialized data with C-Strings it is sufficient to keep the received data
block active and untouched. Using C-Strings can be a good way to keep the
memory impact as low as possible.

If a custom datatype is set for {\tt stringtype} or {\tt bytestype}, the option
{\tt header} can be used to include a custom header file that defines the
relevant datatype. E.g. add {\tt option header="mydatatype.h";} to make sure
the necessary header is included.

\subsection{Data Type Forward Compatibility}
The table below which data types can be changed in the protocol specification
file to which other data types, without breaking binary compatibility.

\begin{tabular}{|l|l|}
\hline
	old data type & potential future data types\\
\hline
	int	& int64\\
	signed	& sint64\\
	unsigned& uint64\\
	int8	& int16..int64 \\
	int16	& int32..int64 \\
	int32	& int64\\
	uint8	& uint16..uint64\\
	uint16	& uint32..uint64\\
	uint32	& uint64\\
	sint8	& sint16..sint64\\
	sint16	& sint32..sint64\\
	sint32	& sint64\\
\hline
\end{tabular}

If all serialized data has non-negative values, {\tt int, int8, int16, int64}
can be changed to their unsigned variants with same or wider bit-width.

Changing to a more narrow bit width is possible, if all serialized data fits
into the new bit width. For code size reasons the deserialization code does not
perform data-width sanity checks. So be sure what you are doing.

All fixed data types currently cannot be upgrade to a higher bit-width.

\section{Extensions over Protocol Buffers}
\subsection{Compiler Options}
\begin{description}

	\item [{\tt Subclasses}]
Option {\tt SubClasses} causes nested messages to be generated as C++
subclasses in their parent message class instead of mangled names that included
the parent message.

\item[{\tt withEqual and withUnequal}]
	This binary option causes generation of comparions operator {\tt =} and
		{\tt !=}, which provides content comparisons of message objects
		of the same class. Set the relevant option to {\tt true} to
		request generation of the relevant comparison operator.

\item[Method Naming]
	All method names can be specified by options that have the same name as
		the default method name. Setting the relevant option ot {\tt
		false} or an empty string suppresses generation of the method
		if possible. Methods that can be renamed using this technique
		are: {\tt toMemory, fromMemory, toWire, toSink}.

\item[{\tt namespace}]
	Use this option to set the {\tt namespace} of the generated message
		classes.  Only the relevant message classes will be put in this
		namespace. The shared library functions will be implemented as
		defined in the library settings.

\item[{\tt optimize}]
	This option can be used to adjust the code generation target. 

\item[{\tt unknown}]
	This option specifies how to deal with unknown data fields. This is for
		forward compatibility of the generated code. In case the
		message description is extended later with additional fields,
		the handling of unknown fields is necessary in the old code to
		be able to deal with data provided by code that is based on a
		later message definition.
	
	The default way of handling unknown data fields is to skip over this
		data, and to simply ignore it. This can be changed to assert on
		unknown data.

\item[{\tt MutableType}]
	To be able to modify the contents of a message field, the mutable
		method is provided. This usually returns a pointer to the
		mutable message member. Using this option the method can be
		modified to return a reference instead of a pointer. This might
		be preferable for certain coding conventions or coding guide
		lines.
	
	Valid values for {\tt mutable\_type} are {\tt pointer}, which is the
		default, and {\tt reference}.

\item[{\tt ErrorHandling}]
	For uncorrupted data transfer there should never be the requirement to
		handle an error. But in the event of data corruption, it might
		be necessary to recover processing, by raising the error to the
		user code. Therefore, different ways for handling the error
		sitation are provided.

Default error handling is to cancel execution by returning an error code.
Additionally it is possible to set the error handling to {\tt assert} or {\tt
throw}. In case of {\tt assert}, an assertion is added to the code that is
exected to not return. For {\tt throw} a throw statement is generated that
passes the integer error code to the catching code. The error code can be used
to identify the location at which the error happened.

\end{description}

\section{Compatibility with Protocol Buffers}
\subsection{ProtoV2 Features Available in WFC}
\subsection{Unsupported Features of Protocol Buffers}
WFC tries to achieve maximum protov2 compatibility with additional extensions
for embedded systems. Therefore, features of protov3 and later are not
implemented in WFC, when there is no obvious benefit for embedded systems.

\end{document}


